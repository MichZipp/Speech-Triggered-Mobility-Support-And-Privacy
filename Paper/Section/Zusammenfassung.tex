\section{Zusammenfassung}
Sprachassistenten wie Amazon Alexa, Apple Siri und Microsoft Cortana glänzen mit einer Vielzahl von Funktionen sowie einer guten Performance. Jedoch bieten diese keine transparenten Privatsphäre für ihre Nutzer. Die im Rahmen dieses Artikels durchgeführte Umfrage zeigte, dass knapp 50\% der Teilnehmer einen Sprachassistenten aktiv nutzen, aber nicht wissen, was mit ihren Daten bei der Nutzung passiert. Den Teilnehmern ist Privatsphäre sowie Datenschutz sehr wichtig, sodass diese bereit wären, Geld für eine transparente Privatsphäre zu bezahlen. 

Anhand der Ergebnisse der Umfrage wurde ein Konzept für ein Sprachassistenten vorgestellt, welches Nutzern transparente Privatsphäre bietet und die Möglichkeit zu bestimmen, ob eine Anwendung auf bestimmte Nutzerdaten zugreifen darf. Für dieses Konzept wurde eine Architektur entwickelt, die aus einer mobilen App, einem Repository und einer Cloud besteht. Dabei stellt das Repository Laufzeitumgebungen für die Cloud und Apps für den Sprachassistenten bereit. Die mobile App dient als Schnittstelle zwischen Nutzer und Cloud. In der Cloud wird die Sprachverarbeitung und die vom Nutzer ausgewählten Apps ausgeführt. Es wurden Technologien vorgestellt, mit denen diese Architektur umgesetzt werden kann. 

Für die Umsetzung und den Betrieb ist eine Infrastruktur notwendig, welche die Sprachverarbeitung durchführt und das nutzerspezifische Modell weitere entwickelt. Eine solche Infrastruktur braucht ausreichend Rechenkapazitäten, die hohe Kosten verursachen können. Für das Repository, welches die Apps verwaltet, muss eine weitere Kontrollinstanz geschaffen werden, die Apps auf datenschutzkonforme Richtlinien überprüft. Das kann aufwendig werden, wenn zahlreiche Apps im Repository veröffentlicht werden sollen. Nach diesem finanziellen Aufwand, welches ein Risiko mit sich bringt, könnte sich zeigen, ob Nutzer wirklich bereit für die Privatsphäre und den Datenschutz zu bezahlen. 

Schlussendlich zeigt dieser Artikel ein Konzept mit Architektur und einer Technologieauswahl, um Nutzern ein Sprachassistenten zu bieten, bei dem der Nutzer bestimmen kann, welche Daten er von sich preisgibt. 