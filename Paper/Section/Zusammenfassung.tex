\section{Zusammenfassung}

Die Umfrage hat eine gute Übersicht über die aktuelle Verwendung von Sprachassistenten gegeben. Zwar ist die Teilnehmerzahl nicht sonderlich groß, auf die Region beschränkt und im Alter nicht ausreichend verteilt, aber ein Stimmungsbild kann dennoch erkannt werden. Die Personen aus der Region Freiburg nutzen mehrheitlich keinen Sprachassistenten. Der Datenschutz ist den Teilnehmern wichtig, wobei mehr als 25\% Geld dafür bezahlen würde. 

 Grundsätzlich ist das Konzept unserer Ansicht nach gelungen, da der Benutzer im Mittelpunkt agiert und je nach Anwendung entscheiden kann, ob Privacy oder Funktionalität überwiegt. Große Anbieter sollten ein ebenfalls zahlungspflichtiges Modell etablieren, in welchem die Nutzer im die Entscheidungen über Privacy und Funktionalität treffen können. 

Die Architektur ist so gestaltet, dass ein solcher Sprachassistent einfach im Betrieb genommen werden kann. Allerdings lässt sich das vorgestellte Konzept nach heutigem Stand schwer umsetzen, da der Nutzer viele Ressourcen für die Verarbeitung benötigen. 

Es sind bereits viele Open-Source-Unterstützungen vorhanden, sodass die Umsetzung erleichtert wird. 