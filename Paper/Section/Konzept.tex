\section{Konzept}\label{sec:konzept}
Zu Beginn wird bei dem Konzept darauf verwiesen, dass keine Rahmenbedingungen im Hinblick auf finanzielle Mittel und Rechenressource angenommen sind.
Eine wichtige Anforderung ist der Datenschutz. Die Entwurfsprinzipien für die mehrseitige Sicherheit von Daten stellt folgende vier Punkte in Vordergrund\cite{kairannenberg}:

\begin{itemize}
\item Datensparsamkeit
\item Kontrollmöglichkeiten für den Nutzer 
\item Auswahlmöglichkeiten und Verhandlungsspielräume 
\item Dezentralisierung und Verteilung

\end{itemize}
Der dritte Punkt \glqq Auswahlmöglichkeiten und Verhandlungsspielräume\grqq{} sind für dieses Konzept wichtig. Durch eine minimale Datenerfassung wird die Funktionalität von Anwendungen eingeschränkt. Der Benutzer soll darüber entscheiden können, für welche Anwendungen die Funktionalität überwiegen und bei welchen Anwendungen der Datenschutz in Vordergrund steht. Die Punkte Datensparsamkeit, Kontrollmöglichkeiten und Dezentralisierung sollen ebenfalls in den Konzeptentwurf mit einfließen.

Für das oder unser Konzept haben wir folgende Punkte identifiziert, welche beim Entwurf von Sprachassistenten berücksichtigt werden sollten:

\begin{itemize}
\item User-Controlled-Privacy
\item Funktionalität
\item Performance
\item Nutzerfreundlichkeit	
\end{itemize}

Bei der User-Controlled Privacy geht es darum, selbst zu entscheiden, ob der Datenschutz oder die Funktionalität überwiegend im Vordergrund stehen. Funktionalität bedeutet für das Konzept auf eine Funktion verzichten zu müssen, welche andere Anbieter bereitstellen. Die Punkte Performance und Nutzerfreundlichkeit sind eng miteinander verbunden. Bei der Performances geht es darum, eine schnelle Systemantwort zu garantieren, sodass eine solches System der Richtlinien der User Experience entspricht. Die Nutzerfreundlichkeit ist noch weiter gefasst. Nicht nur schnelle Antwortzeiten soll das Konzept garantieren, sondern auch bereitgestellte Konfigurationen, die installiert und gestartet werden können. 

Eine hohe Funktionalität kann durch einen dargestellten Personenkontext berücksichtigt werden, wie in Abbildung ?? zu sehen. Durch dieses Modell können Services auf die unterschiedlichen Nutzerbefindlichkeiten Rücksicht nehmen. In einem Beispielszenario \glqq Hobby Notifier\grqq{} kann der Benutzer passende Meldungen erhalten, wenn freie Zeiten in dem Kalender vorhanden sind, das Wetter an dem kommenden Tag gut ist und das Hobby Radfahren sich anbieten würde.



