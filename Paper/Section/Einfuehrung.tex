\section{Einführung}

Die Sprachsteuerung ist eine Interaktionsmöglichkeit, bei der ein technisches Geräte durch die menschliche Sprache bedient wird. Das nächste Lied, der Wecker oder auch Bestellprozesses können damit initiiert werden.  Die Experten gehen von einem wachsenden Markt für die Sprachsteuerungen aus: Die Fachzeitschrift PR Newswire vermutet, dass sich die Spracheinkäufe in den nächsten vier Jahren um das Zwanzigfache ansteigen\cite{prNewswire}. Das Magazin Campaign schätzt, dass die Sprachsuche in Zukunft das Tippen bei der Suche ersetzt\cite{Campaign}. Die Sprachassistenten sind die Schnittstelle zwischen Nutzer und Anwendung, womit die verschiedenen Serviceleistungen verwendet werden können.

Die Serviceleistungen von Apps sind über eine Schnittstelle der Plattform zu erreichen. Die Sprachsteuerung auf der Plattform ist sehr anspruchsvoll, da hier die Spracheingabe viele hochkomplexe Prozessschritte durchläuft. Bisher werden diese Plattformen oftmals nur von großen Konzern angeboten, welche über die finanziellen Mittel und das Knowhow jedesr einzelnen Schrittes verfügen. Universitäten konzentrieren sich meist auf einen dieser Schritte. Es verwundert nicht, dass die Sprachassistenten oftmals von Amazon, Google, Microsoft oder Baidu angeboten werden. Die Open-Source-Varianten sind meist in der Funktionalität und Performance limitiert. Bei den kommerziellen Produkten sind Funktionalität und Performance gut, aber dafür gibt es Bedenken hinsichtlich der Privatsphäre. Mobile Geräte, wie Smartphone und Speaker, senden die Spracheingabe für die Auswertung zu den Rechenzentren. Dabei können Daten erfasst und für andere Zwecke verwendet werden. Die Datenverwendung sind in der Nutzungsbestimmung angegeben, allerdings vermitteln diese eine beschränkte Vorstellung von den möglichen Verwendungsszenarien, wie das Profiling. 

Aus diesem Grund haben wir eine Umfrage mit 110 Teilnehmern durchgeführt, welche die Nutzung von Sprachassistenten erfasst, die Wertschätzung des Datenschutzes ermittelt und dabei die finanzielle Bereitschaft ermittelt. Außerdem gaben die Nutzer Anwendungen an, bei denen ihnen ein hoher Datenschutz wichtig ist. Diese Umfrage ist in Abschnitt ??? angeführt. Aus dem Ergebnis lassen sich Schlussfolgerungen ziehen, welche zum Konzept für eine Sprachassistentenumgebung zusammengefasst wurden. Das Konzept listet Zielanforderungen auf. Anschließend wird das Konzept der User-Controlled-Privacy im Konzeptkontext erläutert.
Nachfolgend wird die hybride Architektur erläutert, die einen Mischung aus Datenschutz und Funktionalität bieten soll. Es wird auf die Ebenen der Architektur, Sprachverarbeitung, Prozesskontrolle und Mobilitätservices eingegangen.
Im nächsten Kapitel wird eine Realisierungsmöglichkeit erklärt. Dabei werden die eingeführten Ebenen mit passenden Unterstützungen ergänzt. 
Eine Beurteilung des Konzeptes, der Technologien und der Umsetzung schließt diese Seminararbeit ab.
