\section{Zusammenfassung}
Sprachassistenten wie Amazon Alexa, Apple Siri und Microsoft Cortana verfügen über eine Vielzahl an Funktionen sowie eine guten Performance. Jedoch bieten diese keine transparente Privatsphäre für ihre Nutzer. Die im Rahmen dieses Artikels durchgeführte Umfrage zeigte, dass knapp 50\% der Teilnehmer einen Sprachassistenten aktiv nutzen, aber nicht wissen, was mit ihren Daten bei der Nutzung passiert. Den Teilnehmern ist Privatsphäre sowie Datenschutz sehr wichtig. Deshalb sind sie bereit für eine bessere Privatsphäre zu bezahlen. 

Anhand der Ergebnisse der Umfrage wurde ein Konzept für einen Sprachassistenten vorgestellt, welcher Nutzern transparente Privatsphäre bietet und die Möglichkeit zu bestimmen, ob eine Anwendung auf bestimmte Nutzerdaten zugreifen darf. Für dieses Konzept wurde eine Architektur entwickelt, die aus einer mobilen App, einem Repository und einer Cloud besteht. Dabei stellt das Repository Laufzeitumgebungen für die Cloud und Apps für den Sprachassistenten bereit. Die mobile App dient als Schnittstelle zwischen Nutzer und Cloud. In der Cloud wird die Sprachverarbeitung und die vom Nutzer ausgewählten Apps ausgeführt. Es wurden Technologien vorgestellt, mit denen diese Architektur umgesetzt werden kann. Somit steht ein Sprachassistent bereit, der dem Nutzer einen transparenten Datenschutz bietet, wobei dieser festlegt, welche Daten genutzt werden dürfen. 

Die Umsetzung der Architektur bringt jedoch auch einige Fragen mit sich. Für die Umsetzung der privaten Cloud können hohe Kosten für ein Nutzer entstehen. Denn für die Nutzung der lokalen Sprachverarbeitung muss die Cloud ausreichend Rechenkapazitäten zur Verfügung stellen. Des Weiteren kann ein hoher Aufwand für die Prüfung der Apps auf Erfüllung der Datenschutzkriterien aufkommen. Auch stellt sich die Frage, ob die Zielgruppe groß genug ist, dass Entwickler Apps für diese beschriebene Architektur entwickeln. Ohne Apps würde der Sprachassistent nicht genutzt werden.

Schlussendlich zeigt dieser Artikel ein Konzept mit Architektur und einer Technologieauswahl, um Nutzern ein Sprachassistenten zu bieten, bei dem der Nutzer bestimmen kann, welche Daten er von sich preisgibt. 