\documentclass[journal]{IEEEtran}
\usepackage{blindtext}
\let\labelindent\relax
\usepackage[inline]{enumitem}
\usepackage{graphicx}
\usepackage[acronym]{glossaries}
\usepackage{subcaption}
\usepackage[bookmarksopen, bookmarksdepth=2, breaklinks=true]{hyperref}

% *** GRAPHICS RELATED PACKAGES ***
%
\ifCLASSINFOpdf
\else
\fi


\newacronym{ble}{BLE}{Bluetooth Low Energy}

\hyphenation{op-tical net-works semi-conduc-tor}


\begin{document}
\title{Speech Triggered Mobility Support And Privacy}

\author{\begin{center}
\begin{tabular}{c c} 
 Marius Becherer & Michael Zipperle \\ 
 \textit{259158} & \textit{259564} \\
 Marius.Becherer@hs-furtwangen.de & Michael.Zipperle@hs-furtwangen.de \\
\end{tabular}
\end{center}}%
        
%\author{Christian Laustsen, \textit{20176018},
%        Anders Rikvold, \textit{20176009},
%        and Michael Zipperle, \textit{20176059}}% <-this % stops a space
%\thanks{M. Shell is with the Department
%of Electrical and Computer Engineering, Georgia Institute of Technology, Atlanta,
%GA, 30332 USA e-mail: (see http://www.michaelshell.org/contact.html).}% <-this % stops a space
%\thanks{J. Doe and J. Doe are with Anonymous University.}% <-this % stops a space
%\thanks{Manuscript received April 19, 2005; revised January 11, 2007.}}

% The paper headers
\markboth{Hochschule Furtwangen - Mobilität und Innovation, Juli 2018}%
{Hochschule Furtwangen - Mobilität und Innovation, Juli 2018}

% make the title area
\maketitle


\begin{abstract}
%\boldmath
Orderings systems for restaurants offer a great way to offload work from restaurants, in taking orders, and lessens the burden of customers needing to acquire and interact with a waiter. In this paper we investigate the feasibility of an offline ordering system for restaurants that only uses commodity hardware (smartphones) and communicates entirely over BLE. This stands in contrast to current systems, which either require special hardware at the tables or that the customer uses a smartphone application that functions over the Internet.

Using technology such as BLE and NFC we propose a system to handle the interaction between the restaurant and customer, along with a prototype implementation, \textit{SmartOrder}, that demonstrates the feasibility of the system. SmartOrder shows that common concerns such as the range and throughput of \gls{ble} are either not a concern or at least acceptable. The prototype implementation achieves a range of 71 meters with no obstacles and 12 meters around corners (6 meters on each side). Data throughput is shown to be somewhat slow, but this is acceptable since data transfers happen infrequently. Finally, both of these limitations are addressed with potential solutions.

\end{abstract}

% Note that keywords are not normally used for peerreview papers.
%\begin{IEEEkeywords}
%IEEEtran, journal, \LaTeX, paper, template.
%\end{IEEEkeywords}

% For peerreview papers, this IEEEtran command inserts a page break and
% creates the second title. It will be ignored for other modes.
\IEEEpeerreviewmaketitle


% *** START OF SECTIONS ***--------------------------------------------

\section{Introduction}
Imagine you have just arrived in a restaurant, and taken a seat at a table. In order to start eating, you first ask the waiter for the menu, which he brings to you. After thinking about what you would like, you call the waiter to your table again, and tell him what you would like to eat. The waiter writes down your order, and takes it back to the restaurant kitchen, where your meal is prepared. Then, hopefully after not waiting too long, the waiter brings you your meal. 

% *** END OF SECTIONS ***---------------------------------------------


% Can use something like this to put references on a page
% by themselves when using endfloat and the captionsoff option.
\ifCLASSOPTIONcaptionsoff
  \newpage
\fi

\begin{thebibliography}{1}
\bibitem{BLEEnergy}
Siekkinen M., et al. Wireless Communications and Networking Conference Workshops (WCNCW), 2012 IEEE.
"How Low Energy is Bluetooth Low Energy? Comparative Measurements with ZigBee/802.15.4"


\end{thebibliography}

\end{document}


