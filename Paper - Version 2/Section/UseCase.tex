\subsection{Anwendungsbeispiel}
Das Anwendungsbeispiel soll aufzeigen, wie durch einen Sprachassistenten die sprachbasierte Mobilitätsunterstützung gefördert und gleichzeitig ein feinkörniger Datenschutz für den Nutzer gewährleistet werden kann. Dabei werden durch das Anwendungsbeispiel folgende Funktionalitäten abgedeckt:

\begin{itemize}
    \item Ärzte in der Umgebung ermitteln.
    \item Einen Arzttermin aushandeln, wobei der Kalender eines Arztes mit dem Kalender des Nutzer abgeglichen wird.
    \item Optionale Mobilitätsunterstützung anfordern:
    \begin{itemize}
        \item Scala Mobile, Unterstützung des Nutzer beim Verlassen des Hauses. Kann notwendig sein, wenn der Nutzer beispielsweise nicht mehr alleine Treppen gehen kann.
        \item Abholservice, der den Nutzer zum Arzt und wieder nach Hause befördert.
    \end{itemize}
\end{itemize}

Ein Nutzer kann die Daten, die für diese Funktionalitäten notwendig sind, detailliert in einer mobilen App festlegen. Beispielsweise kann ein Nutzer seinen Standort automatisch via GPS ermitteln lassen, jedoch kann er diesen genauso in einem Textfeld eingeben. Der Nutzer kann seinen Kalender freigeben, jedoch sind für das Anwendungsbeispiel keine Termindetails notwendig. Es reicht aus zu wissen, ob ein Nutzer zu einem bestimmten Zeitpunkt verfügbar ist oder nicht. Aus diesem Grund kann ein Nutzer Termindetails, wie Titel und Beschreibung eines Termins ausblenden. Mehr Details zur Schützung der Privatsphäre werden im Laufe des Kapitel aufgezeigt. 