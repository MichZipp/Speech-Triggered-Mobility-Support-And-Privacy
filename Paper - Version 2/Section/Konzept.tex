\section{Konzept}\label{sec:konzept}
Anhand der Umfrage wurde ein Konzept für einen Sprachassistenten entwickelt. Dabei wurden die Kosten für benötigte Ressourcen vernachlässigt. Eine wichtige Anforderung an das Konzept ist der Datenschutz. Die Entwurfsprinzipien für die mehrseitige Sicherheit von Daten nach Kai Rannenberg stellt folgende vier Punkte in den Vordergrund \cite{kairannenberg}:

\begin{enumerate}
	\item Datensparsamkeit
	\item Kontrollmöglichkeiten für den Nutzer 
	\item Auswahlmöglichkeiten und Verhandlungsspielräume 
	\item Dezentralisierung und Verteilung
\end{enumerate} 

Im Rahmen dieses Konzepts erfolgt die Fokussierung auf die ersten drei Punkte. Oftmals erfassen Anwendungen Daten eines Nutzers, die nicht der Verbesserung der Anwendungen dienen, sondern zur Analyse der Nutzer und dem Weiterverkauf verwendet werden. Deshalb soll durch das Konzept sichergestellt werden, dass eine Anwendung nur Daten von Nutzern bezieht, die diese auch tatsächlich benötigen. Des Weiteren sollen die erfassten Daten den Nutzern transparent dargestellt werden. Somit können Nutzer ihre erfassten Daten manipulieren bzw. anonymisieren. 

Daraus ergibt sich für den zu entwickelnden Sprachassistenten folgende Anforderungen:
\begin{itemize}
	\item Nutzergesteuerte Privatsphäre
	\item Funktionalität
	\item Performance
	\item Benutzerfreundlichkeit	
\end{itemize}

Durch die nutzergesteuerte Privatsphäre kann ein Nutzer bestimmen, welche Daten er von sich für bestimmte Anwendungen freigibt. Anwendungen benötigen jedoch zusätzliche Daten eines Nutzers, um dadurch Benutzerfreundlichkeit zu bieten. Ein Beispiel ist die Frage eines Sprachassistenten nach der Wettervorhersage. Weiß der Sprachassistent wo sich ein Nutzer aktuell befindet, so kann dieser die Wettervorhersage für die Position des Nutzer liefern. Sonst müsste der Sprachassistent zuerst den Nutzer fragen, für welchen Ort dieser eine Wettervorhersage wünscht. Will ein Nutzer seine Daten für eine Anwendung nicht freigeben, so kann er einen fiktiven Kontext festlegen. Damit kann der Nutzer diese Anwendung nutzen, jedoch auf Kosten der Benutzerfreundlichkeit.