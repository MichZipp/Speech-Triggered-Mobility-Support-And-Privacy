\documentclass[journal]{IEEEtran}
\usepackage{blindtext}
\let\labelindent\relax
\usepackage[inline]{enumitem}
\usepackage{graphicx}
\usepackage[acronym,toc,shortcuts]{glossaries}
\usepackage{subcaption}
\usepackage{float}

% Abstand nach einer Abbildung verringern
\setlength{\belowcaptionskip}{-5pt}


\usepackage{url}
\usepackage{breakurl}
\def\UrlBreaks{\do\/\do-}

\usepackage[bookmarksopen, bookmarksdepth=2, breaklinks=true]{hyperref}
\usepackage[official]{eurosym}
\usepackage{listings}
\usepackage{multicol}
\usepackage[utf8]{inputenc}  
\usepackage[german]{babel}
% *** GRAPHICS RELATED PACKAGES ***
%
\ifCLASSINFOpdf
\else
\fi


\newacronym{stt}{STT}{Speech to Text}
\newacronym{tts}{TTS}{Text to Speech}
\newacronym{nlu}{NLU}{Natural Language Unterstanding}
\newacronym{nlc}{NLC}{Natural Language Classification}
\newacronym{sdk}{SDK}{Software Development Kit}
\newacronym{cmu}{CMU}{Carnegie Mellon University}
\newacronym{iso}{ISO}{Disk Image Optical}
\newacronym{aws}{AWS}{Amazon Web Services}
\newacronym{ki}{KI}{künstlichen Intelligenz}

\hyphenation{op-tical net-works semi-conduc-tor}


\begin{document}
\title{Speech Triggered Mobility Support And Privacy}

\author{\begin{center}
\begin{tabular}{c c} 
 Marius Becherer & Michael Zipperle \\ 
 Hochschule Furtwangen &  Hochschule Furtwangen\\ 
 \textit{259158} & \textit{259564} \\
 Marius.Becherer@hs-furtwangen.de & Michael.Zipperle@hs-furtwangen.de \\
\end{tabular}
\end{center}}%
       
%\thanks{M. Shell is with the Department
%of Electrical and Computer Engineering, Georgia Institute of Technology, Atlanta,
%GA, 30332 USA e-mail: (see http://www.michaelshell.org/contact.html).}% <-this % stops a space
%\thanks{J. Doe and J. Doe are with Anonymous University.}% <-this % stops a space
%\thanks{Manuscript received April 19, 2005; revised January 11, 2007.}}

% The paper headers
\markboth{Hochschule Furtwangen - Mobilität und Innovation, Juli 2018}%
{Hochschule Furtwangen - Mobilität und Innovation, Juli 2018}

% make the title area
\maketitle

\begin{abstract}
%\boldmath 
Aktuelle Sprachassistenten werden von großen IT-Konzernen wie Google, Amazon, Microsoft, Apple oder Baidu angeboten. Die Sprachassistenten umfassen zahlreiche Funktionalitäten, welche i.d.R. zentral in der Cloud von den Anbietern ausgeführt werden. Doch was passiert mit den Eingabedaten der Nutzer? Die Anbieter machen dazu ungenaue Angaben. Die Nutzer können sich nicht sicher sein, ob ihre Privatsphäre und Daten geschützt sind. Es stellt sich die zentrale Forschungsfrage, was mit der sprachbasierten Interaktion zwischen Nutzern und Diensten aktuell geschieht, und welche Konzepte für einen konfigurierbaren Datenschutz durch die Nutzer zukünftig vorstellbar sind. Im Artikel werden die Umfrageergebnisse vorgestellt, die von Sprachassistent-Nutzern im Rahmen dieser Forschungsarbeit ermittelt wurden. Das Ergebnis zeigt insbesondere die Zahlungsbereitschaft für einen individuell konfigurierbaren Datenschutz. Basierend hierauf wird das Konzept für einen konfigurierbaren Sprachassistenten und dessen Architektur präsentiert. Des Weiteren werden Technologien zur zukünftigen Umsetzung dieser Architektur vorgeschlagen.

\end{abstract}

% Note that keywords are not normally used for peerreview papers.
%\begin{IEEEkeywords}
%IEEEtran, journal, \LaTeX, paper, template.
%\end{IEEEkeywords}

% For peerreview papers, this IEEEtran command inserts a page break and
% creates the second title. It will be ignored for other modes.
\IEEEpeerreviewmaketitle


% *** START OF SECTIONS ***--------------------------------------------

\section{Introduction}
Voice control is an interaction facility where technical devices can be controlled by human speech. The next song, the alarm clock or even order processes can be initiated with it. Experts predict a growing market for voice controls: The magazine "PR Newswire" estimates, that purchases over voice will increase twentyfold over the next four years \cite{prNewswire}. The magazine "Campaign" estimates that in the future the search in browsers will be replaced by the search via language \cite{Campaign}. The voice assistants include such voice control services and thus form the interface between users and applications.

The applications of a voice assistants are called apps and run on a platform in the cloud. Speech processing on the platform is sophisticated, as a user's speech input goes through highly complex speech processing sub-processes until an appropriate response can be generated for the user. Currently these platforms are offered by large cloud providers who have the financial resources and the know-how of each sub-process. Universities usually focus on a sub-process. voice assistants are offered by Amazon, Google, Microsoft or Baidu, with many features and good performance. However, there are privacy concerns, as the cloud providers privacy policies do not make it clear what happens to users' data in the cloud. Mobile devices such as smartphones and speakers send the voice input of a user for evaluation to the appropriate cloud provider. Data can be collected and possibly abused. The data usage is stated in the terms of use, but this gives a limited indication of the possible usage scenarios, such as the profiling of users.

For this reason, a survey was conducted with 110 participants. This includes the use of voice assistants, the relevance of data protection from the user's point of view and the financial readiness for more privacy. In doing so, users indicated applications where privacy is particularly important to them. The results are explained in chapter \ref{sec:motivaiton} and are the motivation for the development of the concept of a voice assistant with more privacy for the user. In chapter \ref{sec:konzept} a concept is presented by the user full control about their data. To fulfill this concept, an architecture has been developed for a voice assistant, which is discussed in chapter \ref{sec:architecure}. Afterwards, in chapter \ref{sec:umsetzung} technologies are presented, with which the architecture can be implemented. An assessment of the concept, the technologies and the implementation complete this article. \newline


\section{Related Work}
Sprachassistenten wie Amazon Alexa \cite{alexaAssitent}, Google Assistant \cite{googleAssistant}, Apple Siri \cite{siriAssistent}, Microsoft Cortana \cite{cortanaAssistent} und Baidu DuerOS \cite{baiduAssistant} dominieren aktuell den Markt und sind in Abbildung \ref{fig:sprachassistenten} dargestellt. 

\begin{figure}[h!]
	\centering
	\includegraphics[width=0.9\linewidth]{Picture/Sprachassistenten}
	\caption[Sprachassistenten auf dem Markt\cite{homeAssistants}]{Sprachassistenten auf dem Markt\cite{homeAssistants}}
	\label{fig:sprachassistenten}
\end{figure}

Dabei gehen die Sprachassistenten unterschiedlich mit den Eingabedaten der Nutzer um. Aus den Datenschutzrichtlinien der Anbieter lassen sich keine exakten Informationen finden, was im Detail mit den Eingabedaten eines Nutzers geschieht. Im Folgenden werden die Datenschutzrichtlinien der einzelnen Anbieter kurz beschrieben.

Amazons Alexa verwendet alle Nutzereingaben, um den Sprachservice zu verbessern und personalisierte Werbung anzuzeigen. Eine Beschränkung der Datennutzung für verschiedene Bereiche ist möglich, wodurch sich allerdings auch die Funktionalität einschränkt \cite{alexaPrivacy}.

Google Assistant verwendet die gleichen Berechtigungen, welche für die mobile App des Anbieters gelten \cite{googleShare}. Eine abweichende Einstellung zur mobilen App ist dabei nicht möglich. Die Interaktion mit Google Assistant kann für die personalisierte Werbung genutzt werden, wie sonstige Suchanfragen \cite{googlePrivacy}.

Bei Siri müssen Dienste aktiviert sein, um darauf zurückgreifen zu können. Um die Aussprache und die Funktionalität zu verbessern, werden Daten wie Name, Kontakte, Musik, Suchaktivitäten und weitere Informationen verschlüsselt übertragen. Die Daten werden nicht mit der Apple ID genutzt, sondern mit zufällig erstellter Kennung. Dadurch wird die Privatsphäre für Nutzer gewährleistet \cite{siriPrivacy}.

In den Datenschutzeinstellungen von Microsofts Cortana wird darüber informiert, dass bestimmte Daten \glqq [...] wie z. B. ihre Suchen, Kalender, Kontakte und Orte. [...]\grqq{}\cite{cortanaAssistent} gespeichert werden. Die Datennutzung von Cortana als Personal Assistant ist konfigurierbar. Sind die personalisierten Informationen deaktiviert, kann Cortana nur für Anwendungen wie der Suche und das Festlegen eines Timers genutzt werden. Cortana verwendet personenbezogene Daten nicht für personalisierte Werbung. 

Baidu DuerOS sammelt ebenfalls Nutzerdaten, um die Sprachverarbeitung des Sprachassistenten zu verbessern. Qi Lu verweist auf die vielen Szenarien in denen Baidu Daten sammelt, womit Baidu der Sprung an die Weltspitze im Bereich Künstliche Intelligenz gelingen soll . Die persönlichen Daten eines Nutzers werden übermittelt. Dabei bietet Baidu keine konfigurierbare Privatsphäre an \cite{baiduAI}. 

Seit Februar gibt es den Sprachassistenten Mycroft Mark II, indem Offenheit und Privatsphäre vereint werden \cite{mycroftsmartspeaker}. Die Funktionalitäten sind hier begrenzt, da der Sprachassistent keine Daten speichert um den Benutzerkontext weiter zu trainieren und zu verstehen.
\section{Motivation}\label{sec:motivaiton}
Zu Beginn wurde durch eine Meinungsumfrage überprüft, ob bei Sprachassistenten mehr Datenschutz gewünscht ist. Dabei haben sich 110 Teilnehmer an der Umfrage beteiligt. Die Teilnehmer umfassten folgende Altersgruppen:
\begin{itemize}
	\item 1 bis 18 Jahre 
	\item 19 bis 25 Jahre
	\item 26 bis 35 Jahre
	\item 36 und älter	
\end{itemize}

55,5 \% der Teilnehmer waren männlich und 45,5\%  weiblich, wie in Abbildung \ref{fig:umfrage_teilnehmer} ersichtlich.

\begin{figure}[!h]
	\centering
	\includegraphics[width=0.9\linewidth]{Picture/umfrage_teilnehmer}
	\caption[Teilnehmer der Umfrage]{Teilnehmer der Umfrage}
	\label{fig:umfrage_teilnehmer}
	
\end{figure}

Den Teilnehmern wurden folgende Fragen gestellt:

\begin{enumerate}	
	\item Wie oft nutzen sie einen Sprachassistenten?
	\item Wissen sie was mit ihren Daten geschieht?
	\item Würden sie Geld für eine hohe Datensicherheit bezahlen?
	\item Wie viel Geld würden sie einmalig für eine hohe Datensicherheit einer Anwendung bezahlen?
	\item Bei welchen Anwendungen ist ihnen Privatsphäre besonders wichtig?	
\end{enumerate}

Bei der ersten Frage stellte sich heraus, dass 44,5\% einmal im Monat oder häufiger einen Sprachassistenten in Anspruch nehmen. In den USA wurde eine Studie von \glqq highervisibility\grqq{} durchgeführt, bei der mehr als 70\% der Teilnehmer einen Sprachassistenten mehr als einmal im Monat verwenden \cite{highervisibility}.
Ein direkter Vergleich der Umfrage mit der Studie aus der USA ist in Abbildung \ref{fig:umfrage_haeufigkeit} zu sehen.

\begin{figure}[!h]
	\centering
	\includegraphics[width=0.9\linewidth]{Picture/umfrage_haeufigkeit}
	\caption[Nutzungshäufigkeit von Sprachassistenten]{Nutzungshäufigkeit von Sprachassistenten}
	\label{fig:umfrage_haeufigkeit}
\end{figure}

In der Studie wurden Menschen aus verschiedenen Altersgruppen und Herkunftsländern befragt. Die im Rahmen dieses Artikels durchgeführte Umfrage wurde überwiegend von jungen Leuten beantwortet.

\begin{figure}[!h]
 	\centering
 	\includegraphics[width=0.5\linewidth]{Picture/umfrage_datenschutz}
 	\caption[Relevanz des Datenschutzes für die Umfrageteilnehmer]{Relevanz des Datenschutzes für die Umfrageteilnehmer}
 	\label{fig:umfrage_datenschutz}
\end{figure}

\newpage

Wie in Abbildung \ref{fig:umfrage_datenschutz} zu sehen ist, wissen 90\% der Teilnehmer nicht, was mit ihren Daten passiert. Die Zahlungsbereitschaft für Datensicherheit ist nach Altersgruppen in Abbildung \ref{fig:umfrage_geld_gruppen} visualisiert. Jeder Vierte würde für eine bessere Datensicherheit bezahlen und 56\% der Teilnehmer sind unsicher, ob sie dafür Geld ausgeben würden. Die Altersbetrachtung nach der Zahlungsbereitschaft ist bei der Gruppe unter 18 Jahren am geringsten. Die Schnittmenge der Teilnehmer, welche \glqq Ja\grqq{} oder \glqq Vielleicht\grqq{} angekreuzt haben, steigt mit zunehmendem Alter.

\begin{figure}[!h]
	\centering
	\includegraphics[width=0.9\linewidth]{Picture/umfrage_geld_gruppen}
	\caption[Zahlungsbereitschaft der Teilnehmer in verschiedenen Altersgruppen]{Zahlungsbereitschaft der Teilnehmer in verschiedenen Altersgruppen}
	\label{fig:umfrage_geld_gruppen}
\end{figure}

Der Betrag, welche die Teilnehmer für Anwendungen ausgeben würden variiert sehr und ist in Abbildung \ref{fig:umfrage_betrag} einzusehen. Ungefähr 15\% der befragten Teilnehmer sind nicht bereit dafür zu zahlen, während ein Großteil diese Bereitschaft hat.

\begin{figure}[!h]
	\centering
	\includegraphics[width=0.9\linewidth]{Picture/umfrage_betrag}
	\caption[Zahlungsbereitschaft der Teilnehmer nach Betrag]{Zahlungsbereitschaft der Teilnehmer nach Betrag}
	\label{fig:umfrage_betrag}
\end{figure}

Wie Abbildung \ref{fig:umfrage_anwendung} zeigt, ist den Teilnehmern die Privatsphäre im Bereich Banking, Haussteuerung, Handysteuerung, Soziale Netzwerke und Chatting besonders wichtig.

Somit lassen sich folgende Schlussfolgerungen aus der Umfrage ziehen:
\begin{itemize}	
	\item Sprachassistenten werden in unterschiedlichem Umfang genutzt
	\item Die Nutzer wissen nicht, was mit ihren Daten geschieht
	\item Nutzer würden für den Schutz ihrer Daten bezahlen
	\item Datenschutz ist in den Bereichen Banking, Chatting, Haussteuerung, Social Media und Handysteuerung wichtig.
\end{itemize}

\begin{figure}[!h]
	\centering
	\includegraphics[width=0.9\linewidth]{Picture/umfrage_anwendung}
	\caption[Datenschutzrelevante Anwendungen der Umfrageteilnehmers]{Datenschutzrelevante Anwendungen der Umfrageteilnehmer}
	\label{fig:umfrage_anwendung}
\end{figure}

\section{Konzept}\label{sec:konzept}

\subsection{Anforderungen}

Zu Beginn wollten wir durch eine Meinungsumfrage überprüfen, ob bei Sprachassistenten mehr Datenschutz gewünscht ist. Dabei haben sich 110 Teilnehmer an der Umfragen t beteiligt. Wir unterteilten die Teilnehmer in folgende Altersgruppen:

\begin{itemize}
	\item 0 bis 18 Jahre 
	\item 19 bis 25 Jahre
	\item 26 bis 35 Jahre
	\item 36 und älter	
\end{itemize}

\begin{figure}[h!]
	\centering
	\includegraphics[width=0.7\linewidth]{Picture/umfrage_teilnehmer}
	\caption[Architektur Übersicht]{Architektur Übersicht}
	\label{fig:umfrage_teilnehmer}
\end{figure}

55,5 \% der Teilnehmer sind männlich und 45,5\% sind weiblich. Den Teilnehmern wurden folgende Fragen gestellt:

\begin{enumerate}
	
	\item Wie oft nutzen Sie ein Sprachassistent?
	\item Wissen Sie was mit Ihren Daten passiert?
	\item Würden Sie Geld für eine hohe Datensicherheit bezahlen?
	\item Wie viel Geld würden Sie für eine hohe Datensicherheit einer Anwendung bezahlen (einmalige Zahlung)?
	\item Bei welchen Anwendungen ist Ihnen Privatsphäre besonders wichtig?
	
\end{enumerate}

Bei der ersten Frage stellte sich heraus, dass mehr 54,5\% keine Sprachassistenten verwenden. Die bedeutet, dass 44,5\% einmal in Monat oder häufiger einen solchen Service in Anspruch nehmen. In den USA wurde eine Studie von highervisibility durchgeführt, bei der mehr als 70\% der Teilnehmer einen Sprachassistenten einmal im Monat oder häufiger verwenden\cite{highervisibility}.

\begin{figure}[h!]
	\centering
	\includegraphics[width=0.7\linewidth]{Picture/umfrage_haeufigkeit}
	\caption[Architektur Übersicht]{Architektur Übersicht}
	\label{fig:umfrage_haeufigkeit}
\end{figure}

Bei den ausgewählten Testpersonen verwenden im Vergleich zur Umfrage, weniger Personen eine Sprachsteuerung. In den USA sind Menschen in verschiedenen Altersgruppe und regionaler Herkunft befragt worden. Die durchgeführte Umfrage hatte überwiegend junge Leute befragt. Erwartet wurde eine höhere Nutzung der Sprachsteuerung. 


Ungefähr 90\% der Teilnehmer haben angegeben, dass sie nicht wissen, was mit ihren Daten passiert. Jeder Vierte würde dabei für eine hohen Datensicherheit Geld bezahlen und 56\% der Teilnehmer sind sich unsicher, ob diese dafür Geld bezahlen würden. Die Altersbetrachtung nach der Zahlungsbereitschaft zeigt, dass die Gruppe unter 18 Jahren weniger bereit ist, Geld zu bezahlen. Die Schnittmenge der Teilnehmern, welche Ja oder vielleicht angekreuzt haben, steigt mit zunehmendem Alter.

\begin{figure}
	\centering
	\includegraphics[width=0.5\linewidth]{Picture/umfrage_datenschutz}
	\caption[Relevanz des Datenschutzes für die Umfrageteilnehmer]{Relevanz des Datenschutzes für die Umfrageteilnehmer}
	\label{fig:umfrage_datenschutz}
\end{figure}



\begin{figure}[h!]
	\centering
	\includegraphics[width=0.7\linewidth]{Picture/umfrage_geld_gruppen}
	\caption[Zahlungsbereitschaft der Teilnehmer in verschiedenen Altersgruppen]{Zahlungsbereitschaft der Teilnehmer in verschiedenen Altersgruppen}
	\label{fig:umfrage_geld_gruppen}
\end{figure}

Die Beträge welche die Teilnehmer für eine Anwendung bezahlen, bei den ihnen eine hohe Datensicherheit gewährleistet wird, variiert sehr. Hier wären ungefähr 15\% der Teilnehmer nicht bereit für eine bestimmte Anwendung Geld zu zahlen. 85\% sind bereit für eine Anwendung Geld zu bezahlen, bei welcher ihnen ein hoher Datenschutz gewährleistet wird. 

\begin{figure}[h!]
	\centering
	\includegraphics[width=0.7\linewidth]{Picture/umfrage_betrag}
	\caption[Zahlungsbereitschaft der Teilnehmer nach Betrag]{Zahlungsbereitschaft der Teilnehmer nach Betrag}
	\label{fig:umfrage_betrag}
\end{figure}

Den Teilnehmern ist die Privatsphäre im Bereich Banking, Haussteuerung, Handysteuerung, Soziale Netzwerke und Chatting besonders wichtig. 


\begin{figure}[h!]
	\centering
	\includegraphics[width=0.7\linewidth]{Picture/umfrage_anwendung}
	\caption[Datenschutzrelevante Anwendungen der Umfrageteilnehmers]{Datenschutzrelevante Anwendungen der Umfrageteilnehmer}
	\label{fig:umfrage_anwendung}
\end{figure}

Aufgrund dieser Umfrageergebnisse sind wir zu folgenden Schlussfolgerung gekommen:
\begin{itemize}

\item Personen nutzen teilweise Sprachassistenten
\item Die Nutzer wissen nicht, was mit ihren Daten passiert
\item Nutzer würden für bestimmte Anwendungen Geld bezahlen, wenn diese ihre Daten schützt
\item Datenschutz ist in den Bereichen Banking, Chatting, Haussteuerung, Social Media und Handysteuerung wichtig
	
	
\end{itemize}

\subsection{User-Controlled Privacy}

Zu Beginn wird bei dem Konzept darauf verwiesen, dass keine Rahmenbedingungen im Hinblick auf finanzielle Mittel und Rechenressource angenommen sind.
Eine wichtige Anforderung ist der Datenschutz. Die Entwurfsprinzipien für die mehrseitige Sicherheit von Daten stellt folgende vier Punkte in Vordergrund\cite{kairannenberg}:

\begin{itemize}
\item Datensparsamkeit
\item Kontrollmöglichkeiten für den Nutzer 
\item Auswahlmöglichkeiten und Verhandlungsspielräume 
\item Dezentralisierung und Verteilung

\end{itemize}
Der dritte Punkt „Auswahlmöglichkeiten und Verhandlungsspielräume“ sind für dieses Konzept wichtig. Durch eine minimale Datenerfassung wird die Funktionalität von Anwendungen eingeschränkt. Der Benutzer soll darüber entscheiden können, für welche Anwendungen die Funktionalität überwiegen und bei welchen Anwendungen der Datenschutz in Vordergrund steht. Die Punkte Datensparsamkeit, Kontrollmöglichkeiten und Dezentralisierung sollen ebenfalls in den Konzeptentwurf mit einfließen.

Für das oder unser Konzept haben wir folgende Punkte identifiziert, welche beim Entwurf von Sprachassistenten berücksichtigt werden sollten:

\begin{itemize}
\item User-Controlled-Privacy
\item Funktionalität
\item Performance
\item Nutzerfreundlichkeit	
\end{itemize}

Bei der User-Controlled Privacy geht es darum, selbst zu entscheiden, ob der Datenschutz oder die Funktionalität überwiegend im Vordergrund stehen. Funktionalität bedeutet für das Konzept auf eine Funktion verzichten zu müssen, welche andere Anbieter bereitstellen. Die Punkte Performance und Nutzerfreundlichkeit sind eng miteinander verbunden. Bei der Performances geht es darum, eine schnelle Systemantwort zu garantieren, sodass eine solches System der Richtlinien der User Experience entspricht. Die Nutzerfreundlichkeit ist noch weiter gefasst. Nicht nur schnelle Antwortzeiten soll das Konzept garantieren, sondern auch bereitgestellte Konfigurationen, die installiert und gestartet werden können. 

Eine hohe Funktionalität kann durch einen dargestellten Personenkontext berücksichtigt werden, wie in Abbildung ?? zu sehen. Durch dieses Modell können Services auf die unterschiedlichen Nutzerbefindlichkeiten Rücksicht nehmen. In einem Beispielszenario „Hobby Notifier“ kann der Benutzer passende Meldungen erhalten, wenn freie Zeiten in dem Kalender vorhanden sind, das Wetter an dem kommenden Tag gut ist und das Hobby Radfahren sich anbieten würde.




\section{Architektur}\label{sec:architecure}
\captionsetup[table]{name=Abbildung}
\renewcommand\thetable{11}
\begin{table*}[!h]
	\centering
	\includegraphics[width=1\linewidth]{Picture/Infrastruktur.jpg}
	\caption[Architekturübersicht]{Architekturübersicht}
	\label{fig:infrastruktur}
\end{table*}

Nun gilt es das in Kapitel \ref{sec:konzept} vorgestellte Konzept umzusetzen. Dazu wurde eine Architektur für einen Sprachassistenten entwickelt, welche in Abbildung \ref{fig:infrastruktur} zu sehen ist. 
Diese Architektur bietet mehr Privatsphäre und ist unterteilt in drei Hauptmodule: Die Mobile App, das Repository und die Cloud. Dabei existiert das Modul \glqq Speech Processing\grqq{} in der Mobilen App und in der Cloud. In der Mobilen App werden einfache Sprachverarbeitungsprozesse wie beispielsweise das Aufnehmen bzw. die Wiedergabe eines Audiofiles implementiert. Ressourcenintensive Sprachverarbeitungsprozesse, wie beispielsweise die Umwandlung von Sprache zu Text, werden in die Cloud ausgelagert. Im Folgenden sind die drei Hauptmodule genauer beschrieben.

\subsection{Mobile App}
Die Mobile App ist die Schnittstelle zum Nutzer und wie Abbildung \ref{fig:infrastruktur-app} zeigt, hat diese drei Module:

\begin{description}
	\item \textit{Speech Recording}: Aufnehmen und streamen der Eingabe eines Nutzers an die Cloud.
	\item \textit{Speech Playback}: Abspielen eines Streams, der von der Cloud erzeugt wurde, an den Nutzer.
	\item \textit{Hotword Detection}: Die Mobile App soll nicht ununterbrochen die Eingabe des Nutzers an die Cloud streamen, denn dies könnte die Privatsphäre des Nutzers beeinträchtigen. Die App soll nur aufnehmen und streamen, wenn der Nutzer das will. Somit kommt die sogenannte \glqq Hotword Detection\grqq{} zum Einsatz. Diese belauscht den Nutzer durchgängig lokal auf dem Endgerät. Eine Hotword Detection benötigt kaum Ressourcen, da sie für die Erkennung eines einzigen Signalwortes optimiert wurde. Somit lässt sich ein Signalwort definieren, mit dem das Aufnehmen bzw. streamen an die Cloud gestartet wird. 
\end{description}

\begin{figure}[h!]
	\centering
	\includegraphics[width=0.3\linewidth]{Picture/Infrastruktur-App.jpg}
	\caption[Architektur - Mobile App]{Architektur - Mobile App}
	\label{fig:infrastruktur-app}
\end{figure}

\subsection{Repository}
Das Repository ist in Abbildung \ref{fig:infrastruktur-repository} dargestellt und unterteilt sich nochmals in zwei Module: Das Runtime Repository und das App Repository, welche im folgenden genauer beschrieben werden.

\begin{description}
	\item \textit{Runtime Repository}: Das Runtime Repository stellt eine Laufzeitumgebung für die Cloud bereit. Die Laufzeitumgebung beinhaltet das Betriebssystem und alle vom Sprachassistenten benötigten Pakete und wird in verschieden Formaten bereitgestellt. Ein Nutzer kann eine Laufzeitumgebung im Format seiner Wahl herunterladen und auf seiner privaten Cloud installieren. Dabei soll die Installation möglichst wenig Konfigurationsaufwand erfordern, denn auch Nutzer ohne IT-Background sollen sich diese Umgebungen installieren können. 
	\item \textit{App Repository}: Das App Repository stellt alle Apps bereit, die für den Sprachassistenten verfügbar sind. Will ein Nutzer ein App nutzen, so muss er diese in seiner Laufzeitumgebung installieren. Entwickler können Apps im App Repository für Nutzer bereitstellen. Jedoch müssen diese exakte Angaben über die Nutzung der Daten eines Nutzers tätigen. Des Weiteren wird vor einer Veröffentlichung einer App geprüft, ob die Angaben zum Datenschutz korrekt sind und ob diese erfassten Daten überhaupt für die Verbesserung der Nutzerfreundlichkeit notwendig sind. Somit lässt sich Datensparsamkeit für eine App erreichen.
\end{description}


\begin{figure}[h!]
	\centering
	\includegraphics[width=0.6\linewidth]{Picture/Infrastruktur-Repository.jpg}
	\caption[Architektur - Mobile App]{Architektur - Repository}
	\label{fig:infrastruktur-repository}
\end{figure}

\subsection{Laufzeitumgebung}
Die Laufzeitumgebung ist das Herzstück des Sprachassistenten und ist in Abbildung \ref{fig:infrastruktur-cloud} aufgezeigt. Sie beinhaltet einen großen Umfang an Funktionen.

\setcounter{figure}{11}  
\begin{figure}[h!]
	\centering
	\includegraphics[width=0.9\linewidth]{Picture/Infrastruktur-Cloud.jpg}
	\caption[Architektur - Mobile App]{Architektur - Cloud}
	\label{fig:infrastruktur-cloud}
\end{figure}

\begin{description}
	\item \textit{Sprachverarbeitung}: Das Modul Speech Processing beinhaltet rechenintensive Funktionen der Sprachverarbeitung. Folgende Teilgebiete der Sprachverarbeitung werden für den Sprachassistenten benötigt:
	\begin{itemize}
		\item \ac{stt}: \ac{stt} ist die Umwandlung eines Sprachsignals zu Text.
		\item \ac{tts}: \ac{tts} ist die Umwandlung eines Text zu einem Sprachsignal.
		\item \ac{nlu}: \ac{nlu} ist das Verständnis der natürlichen Sprache. Dies ist nötig, um die Eingabe eines Nutzers zu verstehen. Es kann beispielsweise erkannt werden, ob ein Wort in einem Satz ein Nomen oder Verb ist. 
		\item Language Translation: Hierbei geht es um die Übersetzung eines Texts in eine andere Sprache. Damit kann ein Nutzer mit Nutzern, die eine andere Sprache sprechen kommunizieren. Eine App für den Sprachassistenten kann nicht in der Sprache des Nutzer zur Verfügung stehen, jedoch können die Inhalte der App zur Laufzeit in die Sprache des Nutzers übersetzt werden. Dies minimiert den Entwicklungsaufwand von Apps und erlaubt gleichzeitig das Erreichen einer große Zielgruppe an Nutzern, welche unterschiedlichste Sprachen sprechen. 
		\item Speaker Identifikation: Die Sprecheridentifikation kann als Authentifizierungsmethode des Sprachassistenten dienen. Durch die Sprachinformation in einem Sprachsignal eines Nutzers kann dieser identifiziert werden. Somit kann geprüft werden, ob ein Nutzer die Berechtigung hat, diesen Sprachassistenten zu nutzen.
	\end{itemize}
	\item \textit{Konfigurationsumgebung}: Über die Konfigurationsumgebung lassen sich die Apps- und Privatsphäre-Einstellungen für den Sprachassistenten vornehmen.
	\begin{itemize}
		\item App Management: Dieses Modul verwaltet die vom Nutzer installierten Apps. Dabei können neue Apps aus dem App Repository installiert oder vorhandene Apps deinstalliert werden.
		\item Privacy Management: Dieses Modul verwaltet den Kontext des Nutzers und erlaubt diesem, den Zugriff von Apps auf diesen Kontext festzulegen. Dabei kann bestimmt werden, welche Informationen des Nutzer von einer App zugänglich sind. Des Weiteren lassen sich auch fiktive Kontexte erstellen, sodass der Nutzer ohne Preisgabe seines tatsächlichen Kontexts eine App nutzen kann. Durch dieses Modul kann User-Controlled-Privacy erreicht werden. 
	\end{itemize}	
\end{description}






\section{Umsetzung}
In diesem Kapitel werden Technologien vorgestellt, mit denen die im letzten Kapitel vorgestellt Architektur umgesetzt werden kann. 

\subsection{Mobile App}
Die mobile App wird für verschiedenen mobile Plattformen wie Android, iOS, Windows Phone und als Webanwendung implementiert. Des Weiteren wäre auch die Integration der mobilen App in einen Lautsprecher eine Überlegung Wert. 
\begin{itemize}
	\item \textsl{Speech Recording:} Für das Aufnehmen der Eingabe eines Nutzers ist in den meisten Fällen keine zusätzliches Technologie notwendig. Fast alle mobilen Geräte besitzen bereits ein Mikrofon und die Geräte \ac{sdk}s stellen eine Schnittstelle bereit, mit der auf das Mikrofon zugegriffen werden kann.
	\item \textsl{Speech Playback:} Auch für das Abspielen eines Streams, kann der eingebaute Lautsprecher über die Geräte \ac{sdk} verwendet werden.
	\item \textsl{Hotword Detection:} Im folgenden werden Technologien vorgestellt, mit denen ein Signalworte auf dem mobilen Gerät erkannt werden können: 
	\begin{itemize}
		\item Snowboy Hotword Detection von Kitt.ai: Snowboy Hotword Detection ist ein Apache lizenziertes Software Projekt zur Erkennung von Hotwords. Das Hotword lässt sich frei bestimmen und das Software Projekt ist optimiert für eingebettete Systeme. Laut dem Hersteller Kitt.ai soll die Hotword Detection unter dem kleinsten Raspberry Pi (single-core 700MHz ARMv6) nur 10\% der CPU
		auslasten \cite{SnowboyHotwordDetection}.
		\item Sensory's TrulyHandsfreeTM: Sensory's TrulyHandsfreeTM ist eine Spracherkennung, die für die Erkennung von einzelnen Wörtern bzw. Kommandos optimiert wurde. Zudem zeigt sie sehr gute Ergebnisse in Umgebungen mit vielen Hintergrundger	äuschen auf. Das Vokabular, welches erkannt werden soll, kann durch Sensory's Grammatik Tool erstellt werden. Sensory's TrulyHandsfreeTM ist verfügbar für Android, iOS, QNX, Windows und Mikrocontroller \cite{TrulyHandsfreeTM}.
		\item Pocketsphinx von \acs{cmu} Sphinx: Sphinx ist ein Forschungsprojekt der \ac{cmu}, das sich mit Spracherkennung befasst. Sphinx basiert auf der Open-Source-Lizenz und kann somit frei verwendet werden, solange erkenntlich gemacht wird, dass es sich um eine Software von \ac{cmu} handelt. Pocketsphinx wurde ebenso wie Sensory's TrulyHandsfreeTM auf die Erkennung von einzelnen Wörter optimiert \cite{Pocketsphinx}.
	\end{itemize}
\end{itemize}

\subsection{Repository}
Das Repositorys kann durch einen Webserver umgesetzt werden. Dieser Webserver stellt einerseits verschiedene Laufzeitumgebungen und anderseits Apps für den Sprachassistenten zum Download bereit. Die Laufzeitumgebungen können in den folgenden Formaten angeboten werden:
\begin{itemize}
	\item \textsl{VMWare Image:} Das VMWare Image enthält bereits ein vorinstalliertes Betriebssystem sowie alle nötigen Pakete für den Sprachassistenten. Ein Nutzer muss dieses Image in seine VMWare Umgebung (VMWare vSphere, Workstation, Player) importieren und kleine Netzwerkkonfiguration vornehmen. Diese Umgebung lässt sich am einfachsten Nutzen und benötigt am wenigsten Konfigurationsaufwand. Zudem hat der Nutzer die Möglichkeit, die virtuelle Maschine auf einen anderen Server zu übertragen oder Snapshots zu erstellen. Snapshots sind Abbilder einer virtuellen Maschine, die einen bestimmten Zustand speichern. Dadurch kann bei einer Fehlkonfiguration einfach zum letzten funktionalen Abbild zurückgesprungen werden \cite{VMWare}.
	\item \textsl{Docker Image:} Das Docker Image ist eine Konfigurationsdatei für eine Container Umgebung. Diese Konfigurationsdatei beinhaltet alle zu installierenden Pakete und Konfiguration für den Sprachassistenten. Wird diese Konfigurationsdatei in einer Container Umgebung gestartet, werden die Pakete automatisch installiert und der Sprachassistent konfiguriert. Der Nutzer muss lediglich Netzwerkeinstellung vornehmen, bevor dieser den Sprachassistenten nutzen kann \cite{Docker}.
	\item \textsl{\acs{iso}-Image:} Das \acs{iso}-Image bietet dem Nutzer am meisten Flexibilität und Konfigurationsmöglichkeiten. Der Nutzer kann entscheiden, ob er die Laufzeitumgebung physikalisch oder virtualisiert installieren möchte. Des Weiteren ist dieses nicht an eine Virtualisierungssoftware wie VMWare gebunden und kann virtualisiert auf einer privaten oder öffentlichen Cloud installiert werden. Eine öffentlichen Cloud nimmt dem Nutzer Aufgaben wie Visualisierung, Backups, Ausfallsicherheit und Load Balancing ab, mögliche Anbieter sind \ac{aws} mit Amazon EC2 \cite{AWSAmazonEC2}, Microsoft Azure \cite{MicrosoftAzure} und IBM Bluemix \cite{IBMBluemix}. Das \acs{iso} Image beinhaltet auch alle nötigen Pakete für den Sprachassistenten. Der Nutzer wird bei der Installation des \acs{iso} Images durch ein Wizard geführt, bei dem alle Konfiguration vorgenommen werden.
\end{itemize}

\subsection{Laufzeitumgebung}
\subsubsection{Sprachverarbeitung}
In der Laufzeitumgebung werden Sprachverarbeitungsprozesse durchgeführt, es gibt zwei Möglichkeiten diese Prozesse durchzuführen, entweder lokal auf der Laufzeitumgebung oder durch die Nutzung von sprach-basierten Cloud Services. Im folgenden werden die Vor- bzw. Nachteile dieser Möglichkeiten sowie mögliche Technologien erläutert.
\begin{itemize}
	\item \textsl{Cloudbasierte Sprachverarbeitung:} Die Nutzung von sprach-basierte Cloud Services zur Sprachverarbeitung bietet die bestmögliche Performance. Die Sprachverarbeitung basiert i.d.R. auf einer \ac{ki}, welche sich durch Eingabedaten verbessert. Da Cloud Services von vielen Anwendungen genutzt werden, steht der \ac{ki} eine große Anzahl an Eingabedaten zur Verfügung. Dabei ergibt sich aber der Nachteil, dass die \ac{ki} Eingabedaten eines Nutzers, welche dessen Kontext beschreiben, zur Verbesserung der \ac{ki} nutzen und somit die Privatsphäre nicht optimal geschützt werden kann. Des Weiteren fallen bei der Nutzung von Cloud Services kosten an. Folgende Anbieter bieten sprach-basierte Cloud Services an:
	\begin{itemize}
		\item \ac{aws}: Amazon bietet zahlreiche sprach-basierte Cloud Services an. Amazon Comprehend ist ein Service zum \ac{nlu}, dabei können Einblicke in Zusammenhänge und Beziehungen eines Texte gewonnen werden \cite{AmazonComprehed}. Mit Amazon Translate können Texte, Webseiten und Anwendungen natürlich klingend und akkurat übersetzt werden \cite{AmazonTranslate}. Amazon Transcript kann Sprache zu Text und Amazon Polly Texte zu Sprache umwandeln \cite{AmazonTranscript} \cite{AmazonPolly}. Mit Amazon Lex können Konversationsschnittstellen für Anwendungen erzeugt werden. Dabei dient ein Chatbot als Konversationsschnittstelle und kann auf eine bestimmte Eingabe, die zugehörige Antwort liefern. Amazon Lex nutzt die gleichen Tieflerntechnologien als der Sprachassistent Alexa von Amazon \cite{AmazonLex}.
		\item Microsoft Azure: Microsoft Azure bietet Cognitive Services an, darunter Services zur Sprach zu Text, Text zu Sprache, Übersetzung von Texten und \ac{nlu}. Des Weiteren wird ein Service zur Sprechererkennung und zur Rechtschreibkorrektur angeboten \cite{MicrosoftAzureCognitiveServices}. Gerade die Sprechererkennung ist ein wichtiger Bestandteil für das in diesem Artikel vorgestellten Konzept eines Sprachassistenten, dieser Service kann zur Authentifizierung eines Nutzers genutzt werden. 
		\item IBM Watson: Auch IBM Watson bietet sprach-basierte Cloud Services zur Sprach zu Text und Text zu Sprach Umwandlung an. Des Weiteren wird ein Service für \ac{nlu} und \ac{nlc}, wobei die Absicht einer Eingabe ermittelt wird, angeboten. Mit Watson Assistant kann ein Chatbot realisiert werden \cite{IBMWatsonSpeechServices}.
	\end{itemize}
	\item \textsl{Lokale Sprachverarbeitung:} Der Einsatz einer lokalen Sprachverarbeitung auf der Laufzeitumgebung bietet einer besser Kontrolle der Privatsphäre des Nutzers. Dafür werden auf der Laufzeitumgebung deutlich mehr Ressourcen benötigt und die Performance ist meist schlechter als bei der Nutzung der sprach-basierter Cloud Services. Es gibt einige Anbieter, die Frameworks für die lokale Sprachverarbeitung anbieten. Darunter auch Open-Source-Projekte, wodurch keine Kosten für die Nutzung anfallen würden. 
	\begin{itemize}
		\item Nuance: Nuance arbeitet schon mehr als 25 Jahren an Lösungen für die Sprachverarbeitung, dabei fokussiert sich Nuance auf die Integration dieser Lösungen in mobile Geräte wie Smartphones oder Autos. Unter anderem werden Lösungen zur Sprach zu Text und Text zu Sprach Umwandlungen und zur Erstellung eines Chatbots angeboten \cite{Nuance}. 
		\item Mozilla: Mozilla startet mit dem Common-Voice-Projekt eine Initiative, die dabei helfen soll Maschinen beizubringen, wie echte Menschen sprechen. Diese Initiative befindet sich noch in der Entwicklung, aktuell werden Datensätze zur Verbesserung der \ac{ki} gesammelt. Dabei gilt die Initiative als Open-Source-Projekt und jede Person kann Datensätze zur Verbesserung beisteuern. Dazu hat Mozilla eine Webseite und mobile App erstellt mit der Mitwirkende Datensätze prüfen oder Eingabedaten tätigen können \cite{MozillaCommonVoice}. Welches Potential diese Initiative hat, wird sich erst nach der Beendigung der Entwicklung zeigen.
		\item Kaldi: Kaldi ist ein Spracherkennungs-Toolkit, das unter der Apache-Lizenz frei verfügbar ist. Kaldi zielt darauf ab, Software zu liefern, die flexibel und erweiterbar ist \cite{Kaldi}. Das Projekt wird auf GitHub verwaltet und somit können Entwickler bei der Verbesserung des Toolkits beitragen.
		\item CMUSphinx: CMUSphinx bietet mit Pocketsphinx eine sprecherunabhängige kontinuierliche Spracherkennung an. Dabei ist Pocketsphinx ein Open-Source-Projekt und wird auf GitHub verwaltet. Pocketsphinx kann in mobilen Geräten eingesetzt werden, dabei wird eine Version für Smartwatches angeboten, die keine Internetverbindung benötigt. Bereits trainierte Modelle für die \ac{ki} der Spracherkennung werden angeboten, können jedoch auch selbst trainiert werden \cite{Pocketsphinx}.
	\end{itemize}
\end{itemize}

\subsubsection{Konfigurationsumgebung}
Die Konfigurationsumgebung kann durch eine Web-Anwendungen realisiert werden. Dabei kann der Nutzer über ein beliebiges Gerät mit einem Browser seinen Sprachassistenten konfigurieren. Des Weiteren sollen sich diese Konfigurationen auch mit der mobilen App vornehmen lassen. Abbildung \ref{fig:prototyp} zeigt einen Prototyp, wie diese aussehen könnte. Dabei kann ein Nutzer seinen Kontext einsehen und bestimmen welche Informationen des Kontexts von Apps genutzt werden dürfen. Der Nutzer kann seinen Kontext manipulieren, falls bestimmte Informationen von einer App benötigt werden, er diese aber aus Gründe der Privatsphäre nicht preisgeben will. Somit hat der Nutzer volle Kontrolle über seine Daten und kann festlegen, welche App welche Information beziehen darf. 

\begin{figure}[!ht]
	\centering
	\begin{subfigure}{0.32\linewidth}
		\centering
		\includegraphics[width=1\linewidth]{Picture/App-Store}
		\caption{Store}
		\label{fig:prototyp1}
	\end{subfigure}%
	\begin{subfigure}{0.32\linewidth}
		\centering
		\includegraphics[width=1\linewidth]{Picture/App-Settings}
		\caption{Einstellungen}
		\label{fig:prototyp2}
	\end{subfigure}
	\begin{subfigure}{0.32\linewidth}
		\centering
		\includegraphics[width=1\linewidth]{Picture/App-Kontext}
		\caption{Nutzer Kontext}
		\label{fig:prototyp3}
	\end{subfigure}%
	\caption{Prototyp der App}
	\label{fig:prototyp}
\end{figure}












\section{Zukünftige Arbeit}
Dieser Artikel zeigt das Probleme aktueller Sprachassistenten auf. Mit dem vorgestelltem Konzept kann einem Nutzer eine bessere Privatsphäre gewährleistet werden. Im letzten Kapitel wurden einige Technologien zur Umsetzung dieses Konzeptes vorgestellt. Es zeigte sich, dass es eine große Auswahl an Technologien, gerade im Bereich der Sprachverarbeitung, gibt. Als nächsten Schritt müsste diese Technologien evaluiert werden, um eine Technologieauswahl treffen zu können. Anhand dieser Auswahl kann festgelegt werden, welche Ressourcen die Cloud zur Nutzung dieses Technologien benötigt. Des Weiteren lässt sich anhand der benötigten Ressourcen sowie Technologien ein genaueres Kostenmodell erstellen. Dieses kann genutzt werden, um eine erneute Umfrage bzw. Interviews mit der Zielgruppe durchzuführen. Die Umfrage soll zeigen ob die Zielgruppe bereit ist, die zusätzlichen Kosten für eine bessere Privatsphäre zu bezahlen. Danach kann mit einer prototypischen Entwicklung des Sprachassistenten begonnen werden. Zusätzlich gilt es ein Konzept zu entwickeln, mit welchem Apps auf unerwünschten Datenweitergabe an Dritte geprüft werden können. 
\section{Zusammenfassung}

Durch den entwickelten Prototyp, welcher dem Nutzer Datenschutz und Kontrollmöglichkeiten bietet, konnten einige Rückschlüsse gezogen werden. Dabei konnte die Konzeption bei der Implementierung berücksichtigt werden und Aspekte der mehrseitigen Sicherheit und benutzergesteuerten Privatsphäre finden sich ebenfalls im Prototyp. Der Prototyp besteht aus der Sprachverarbeitungsumgebung (Cloud Services), der Mobile App und dem Privacy Provider.

Anhand der genutzte Cloud Services von Amazon konnten die Funktionalitätsanforderungen erfüllt werden. Aktuell unterstützen die sprachbasierten Cloud Services von Amazon kein Deutsch, deshalb wurde der Prototyp mit Englisch als Sprache umgesetzt. Amazon kündigte jedoch an, bald mehr Sprachen, unter anderem auch Deutsch, anzubieten. Außerdem kann durch die erfüllten DSVGO-Richtlinien der Sprachservices der Datenschutz für die Nutzer gewährleistet werden. Durch die Verwendung der sprachbasierten Cloud Services müssen sich Entwickler nicht detailliert mit der Sprachverarbeitung auseinandersetzen, sondern können auf abstrakter Ebene eine Anwendung entwickeln.  
Bevor ein Nutzer den Sprachassistenten nutzen kann, muss dieser sich authentifizieren. Somit wird der Zugriff vor nichtberechtigten Personen geschützt. In der App können Nutzer verschiedene Profile mit unterschiedlichen Daten anlegen. Hierbei ist die Verwendung von Pseudoprofilen möglich. Im Hinblick auf das Konzept zur mehrschichtigen Sicherheit ist dies wichtig, um Auswahlmöglichkeiten und Verhandlungsspielräume für den Nutzer zu schaffen. Die Nutzerdaten sind in einem Privacy Provider abgelegt. Auch für den Privacy Provider gilt das Konzept der mehrseitigen Sicherheit. Hier ist die Dezentralisierung und Verteilung von großer Bedeutung. Durch die Technologie- und Anbieterauswahl wird auf allen Ebenen der Datenschutz berücksichtigt. Die Daten sind in Deutschland, womit auch eine Rechtslage angewendet wird, die deutlich strenger bei der Datenhaltung ist. Allerdings gibt es beim Privacy Provider auch noch Optimierungspotenzial. Zum einen soll der Ressourcenzugriff konfigurierbarer durch Berechtigung, Dauer und Filterung gestaltet werden. 

In diesem Prototyp wurden nur ein paar Anwendungsfälle umgesetzt. Je nach Nutzer variieren die Anforderungen an einen Sprachassistenten. Verschiedene Anwendungen sollten anhand der Bedürfnisse eines Nutzer aktiviert oder deaktiviert werden können. Dieses Funktionalität könnte den Prototyp in der Zukunft erweitern. Dabei müssen die angebotenen Anwendungen die Nutzer über verwendeten Daten informieren und damit Transparenz  schaffen. Um ein Ökosystem zu schaffen, indem jede Anwendung auf die Daten zugreifen kann, muss ein Standard über die Datenablage geschaffen werden. Andernfalls müssen die Anwendungen die Nutzerdaten selbst verwalten und das Konzept über die Trennung von Daten und Anwendung wäre hinfällig. 

In diesem Artikel wird an verschiedenen Stellen auf das Potenzial von Sprachassistenten verwiesen. Durch die angenehme Bedienung von der Systemintelligenz bietet es einen Mehrwert im Alltag. Allerdings müssen sich verschiedene Branchen öffnen und Schnittstellen anbieten, sodass Buchungen und Reservierungen nicht nur per E-Mail oder Telefon möglich sind. Ist die Infrastruktur von Unternehmen geschaffen, werden Sprachassistenten zusätzlich an Attraktivität für die Nutzer gewinnen.
\section{Danksagung}
Diese Artikel wurde im Rahmen eines Semesterprojekts an der Fachhochschule Furtwangen und unter Betreuung von Prof. Dr. Achim P. Karduck erstellt, dem wir für die gute Betreuung und die vielen Denkanstöße danken.
% *** END OF SECTIONS ***---------------------------------------------


% Can use something like this to put references on a page
% by themselves when using endfloat and the captionsoff option.
\ifCLASSOPTIONcaptionsoff
  \newpage
\fi

\begin{thebibliography}{1}
\bibitem{Campaign}
Christi Olson: "Just say it: The furture of search is voice and personal digital assistant", Campaign 25.04.2016 \url{https://www.campaignlive.co.uk/article/just-say-it-future-search-voice-personal-digital-assistants/1392459},
Zuletzt besucht: 13.07.2018

\bibitem{prNewswire}
OC\&C Strategy Consultants: "Voice Shopping Set to Jump to \$40 Billion By 2022, Rising From \$2 Billion Today", PR Newswire, 28.02.2018, \url{https://www.prnewswire.com/news-releases/voice-shopping-set-to-jump-to-40-billion-by-2022-rising-from-2-billion-today-300605596.html}, 
Zuletzt besucht: 13.07.2018

\bibitem{highervisibility}
Adam Heitzman: "How popular is voice search?", 07.02.2017, higervisibility.com, \url{https://www.highervisibility.com/blog/how-popular-is-voice-search/},
Zuletzt besucht: 13.07.2018

\bibitem{homeAssistants}
”Choosing The Best Voice Assistant For Your Home”, Geeks of Technology,
16.01.2018, \url{https://geeksfl.com/blog/best-voice-assistant/}, Last visit:
20.07.2018 .

\bibitem{kairannenberg}
Rannenberg, Kai. "Mehrseitige Sicherheit—Schutz für Unternehmen und ihre Partner im Internet." Wirtschaftsinformatik 42.6 (2000): 489-497.

\bibitem{cortanaAssistent}
Mircosoft Inc: "Cortana, Ihre persönliche digitale Assistentin", \url{https://privacy.microsoft.com/de-de/windows-10-cortana-and-privacy},
Zuletzt besucht: 05.06.2018

\bibitem{siriAssistent}
Apple Inc.: "Hey Siri, weck mich morgen früh um 7:00 Uhr.",
\url{https://www.apple.com/de/ios/siri/},
Zuletzt besucht: 05.06.2018

\bibitem{alexaAssitent}
Amazon Inc.: "Alexa Assistant"
\url{https://www.amazon.de/b?ie=UTF8&node=12775495031},
Zuletzt besucht: 06.06.2018

\bibitem{alexaPrivacy}
Amazon Inc.: "Alexa Internet Privacy Notice", 23.05.2018,
\url{https://www.alexa.com/help/privacy},
Zuletzt besucht: 06.06.2018

\bibitem{googleAssistant}
Google LLC: "Google Assistant - Just say" ,
\url{https://assistant.google.com/#?modal_active=none},
Zuletzt besucht: 06.06.2018

\bibitem{googlePrivacy}
Google LLC: "Datenschutzerklärung \& Nutzungsbedingungen", 25.05.2018, \url{https://policies.google.com/privacy#whycollect},
Zuletzt besucht: 06.06.2018

\bibitem{baiduAssistant}
Baidu Inc.: "What makes our Artificial Intelligence technology unique"  \url{https://dueros.baidu.com/en/index.html},
Zuletzt besucht: 06.06.2018

\bibitem{baiduPrivacy}
Baidu Inc.: "Baidu Statement of Privacy Protection", \url{http://ir.baidu.com/phoenix.zhtml?c=188488\&p=privacy},
Zuletzt besucht: 06.06.2018


\bibitem{baiduAI}
Jessi hemple: "How Baidu will win china's AI race - and, maybe, the world's", 08.09.2017, \url{https://www.wired.com/story/how-baidu-will-win-chinas-ai-raceand-maybe-the-worlds/},
Zuletzt besucht: 06.06.2018

\bibitem{googleShare}
Google LLC: "Choose what to share with your Google Assistant", \url{https://support.google.com/assistant/answer/7126196?hl=en},
Zuletzt besucht: 06.06.2018

\bibitem{siriPrivacy}
Apple Inc.: "Umgang mit Datenschutz", \url{https://www.apple.com/de/privacy/approach-to-privacy/},
Zuletzt besucht: 06.06.2018

\bibitem{mycroftsmartspeaker}
Jack Wallen: "Mycroft Mark II offers something its digital assistant competitors can't: Privacy and openness", 14.02.2018, \url{https://www.techrepublic.com/article/mycroft-mark-ii-offers-consumers-what-other-digital-assistants-cant-privacy/},
Zuletzt besucht: 06.06.2018	
	
\bibitem{SnowboyHotwordDetection}
Kitt.ai: "Snowboy Hotword Detection",
\url{https://snowboy.kitt.ai/},
Zuletzt besucht: 10.07.2018

\bibitem{TrulyHandsfreeTM}
Sensory: "TrulyHandsfreeTM",
\url{http://www.sensory.com/products/embedded-software-and-sdks/},
Zuletzt besucht: 10.07.2018

\bibitem{VMWare}
"VMWare",
\url{https://www.vmware.com/},
Zuletzt besucht: 11.07.2018

\bibitem{Docker}
"Docker",
\url{https://www.docker.com/},
Zuletzt besucht: 11.07.2018

\bibitem{IBMBluemix}
IBM Bluemix: "Virtual Server",
\url{https://www.ibm.com/cloud/virtual-servers},
Zuletzt besucht: 11.07.2018

\bibitem{AWSAmazonEC2}
Amazon: "Amazon EC2",
\url{https://aws.amazon.com/de/ec2},
Zuletzt besucht: 11.07.2018

\bibitem{MicrosoftAzure}
Microsoft Azure: "Virtual Server",
\url{https://azure.microsoft.com/en-us/services/virtual-machines/},
Zuletzt besucht: 11.07.2018

\bibitem{AmazonComprehed}
Amazon: "Amazon Comprehed",
\url{https://aws.amazon.com/de/comprehend/},
Zuletzt besucht: 11.07.2018

\bibitem{AmazonTranslate}
Amazon: "Amazon Translate",
\url{https://aws.amazon.com/de/translate/},
Zuletzt besucht: 11.07.2018

\bibitem{AmazonTranscript}
Amazon: "Amazon Transcript",
\url{https://aws.amazon.com/de/transcribe/},
Zuletzt besucht: 11.07.2018

\bibitem{AmazonPolly}
Amazon: "Amazon Polly",
\url{https://aws.amazon.com/de/polly/},
Zuletzt besucht: 11.07.2018

\bibitem{AmazonLex}
Amazon: "Amazon Lex",
\url{https://aws.amazon.com/de/lex/},
Zuletzt besucht: 11.07.2018

\bibitem{MicrosoftAzureCognitiveServices}
Microsoft Azure: "Cognitive Services",
\url{https://azure.microsoft.com/en-us/services/cognitive-services/directory/speech/},
Zuletzt besucht: 11.07.2018

\bibitem{IBMWatsonSpeechServices}
IBM Watson: "Speech Services",
\url{https://www.ibm.com/watson/services/},
Zuletzt besucht: 11.07.2018

\bibitem{Nuance}
"Nuance",
\url{https://www.nuance.com},
Zuletzt besucht: 12.07.2018

\bibitem{MozillaCommonVoice}
"Mozilla Common Voice",
\url{https://voice.mozilla.org/de},
Zuletzt besucht: 12.07.2018

\bibitem{Kaldi}
"Kaldi",
\url{http://kaldi-asr.org/},
Zuletzt besucht: 12.07.2018

\bibitem{Pocketsphinx}
CMUSphinx: "Pocketsphinx"
\url{https://cmusphinx.github.io},
Zuletzt besucht: 12.07.2018
\end{thebibliography}

\end{document}


