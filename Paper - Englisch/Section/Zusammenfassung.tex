\section{Conclusion}
Voice assistants like Amazon Alexa, Apple Siri and Microsoft Cortana have a variety of features and good performance. However, these do not provide transparent privacy for their users. The survey conducted under this article showed that nearly 50\% of participants actively use a voice assistant, but do not know what happens to their data during use. The participants privacy and privacy is very important. That's why they are ready to pay for better privacy.

Based on the results of the survey, a concept for a voice assistant was presented that provides users with transparent privacy and the ability to determine if an application is allowed access to specific user data. For this concept, an architecture has been developed that consists of a mobile app, a repository and a cloud. The repository provides runtime environments for the cloud and apps for the voice assistant. The mobile app serves as the interface between the user and the cloud. Voice processing and user-selected apps are running in the cloud. Technologies were presented to implement this architecture. Thus, a voice assistant can be provided, which binds the user to transparent data protection, wherein the user clearly defines which data may be used.

The implementation of the architecture, however, also brings with it some questions. For the implementation of the private cloud, high costs can be incurred by one user. Because for the use of local voice processing, the cloud has to be provided sufficient computing capacity. Furthermore, a high effort for the testing of the apps to meet the data protection criteria arise. The question also arises whether the target group is large enough that developers develop apps for this described architecture. Without apps, the voice assistant would not be used.

Finally, this article presents a concept with architecture and a technology selection to provide users with a voice assistant where the user can determine what data they reveal about themselves.